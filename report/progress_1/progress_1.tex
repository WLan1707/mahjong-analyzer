\documentclass{beamer}
\usetheme{metropolis}           % Use metropolis theme
\usepackage{graphicx}
\usepackage{xcolor}
\usepackage{listings}
\title{Mahjong Analyzer}
\subtitle{Mini Projek}
\date{\today}
\author{Muhammad Jilan Wicaksono}
\institute{Universitas Indonesia}

\lstset{
  language=Haskell,            % Bahasa default
  basicstyle=\ttfamily\small,  % Font dan ukuran
  keywordstyle=\color{blue},   % Warna untuk keyword (misal: "let", "in")
  commentstyle=\color{green!60!black}, % Warna untuk komentar
  stringstyle=\color{red},     % Warna untuk string
  numbers=left,                % Tampilkan nomor baris di kiri
  numberstyle=\tiny\color{gray}, % Style nomor baris
  breaklines=true,             % Otomatis pindah baris jika terlalu panjang
  frame=single,                % Beri bingkai
}
\begin{document}
  \maketitle
  \section{Apa itu Mahjong?}
  \begin{frame}
    % {Apa itu Mahjong?}
    \begin{itemize}
        \item Permainan papan yang menggunakan \((3\times 9 + 7) \times 4\) tile.
    \end{itemize}
    \begin{center}
        \includegraphics[width=.9\linewidth]{image/mahjongTile.jpg}
    \end{center}
  \end{frame}

  \begin{frame}
    \begin{itemize}
        \item Dimainkan oleh \(4 \) atau \(3 \) pemain.
        \item Awalnya, setiap pemain mendapatkan \textit{hand} berisi \(13\) tile.
        \item Tiap giliran, satu pemain mengambil satu tile dari dek (disebut dinding atau \textit{wall}) untuk mendapatkan bentuk tertentu.
    \end{itemize}
    \begin{center}
        \includegraphics[width=.7\linewidth]{image/mahjongInterface.jpg}
    \end{center}
  \end{frame}

  \begin{frame}
    \begin{center}
        \includegraphics[width = .9\linewidth]{image/mahjongMeldType.jpg}
    \end{center}
    \begin{itemize}
        \item Satu hand terdiri dari \(13\) tile.
        \item Ada dua jenis kumpulan tile:
        \begin{itemize}
            \item Pair: \(2 \) tile yang sama.
            \item Meld: \(3 \) tile yang bisa berbentuk Sequence ataupun Triplet.
        \end{itemize}
    \end{itemize}
  \end{frame}

  \begin{frame}
    \begin{center}
        \includegraphics[width = \linewidth]{image/mahjongWinningHand.jpg}
    \end{center}
    \begin{itemize}
        \item Winning hand (\textit{Agari}): hand dengan \(14\) ubin (\(13 + 1\) setelah draw sebelum discard) yang memenuhi pola tertentu.
        \item Biasanya harus berupa \(4\) meld disertai dengan \(1\) pair.
        \item Jika suatu pemain mencapai agari, pemain itu dapat menghentikan ronde tersebut, mendapatkan skor sesuai dengan hand terakhirnya, lalu ronde baru dimulai.
    \end{itemize}
  \end{frame}

  \section{Permasalahan}

  \begin{frame}
    Mudah dilihat bahwa contoh tadi memuat \(4\) meld dan \(1\) pair.
    \begin{center}
        \includegraphics[width = \linewidth]{image/mahjongWinningHand.jpg}
    \end{center}
    Namun, terkadang ada hand yang sulit untuk dilihat, seperti
    \begin{center}
        \includegraphics[width = \linewidth]{image/mahjongComplexHand.jpg}
    \end{center}
    Akan dibuat suatu fungsi untuk memeriksa apakah suatu hand adalah agari.
  \end{frame}

  \begin{frame}
    Satu hal lain yang menarik untuk dihitung adalah seberapa jauh suatu hand dari agari. Hal ini disebut sebagai shanten number.

    contoh \(1 \) shanten
    \begin{center}
        \includegraphics[width = \linewidth]{image/mahjongShanten1.jpg}
    \end{center}

    contoh lain
    \begin{center}
        \includegraphics[width = \linewidth]{image/mahjongShanten2.jpg}
    \end{center}
  \end{frame}

  \begin{frame}
    Yang terakhir adalah menghitung skor yang diterima dari agari
    
    \begin{center}
        \includegraphics[width = \linewidth]{image/mahjongScore.jpg}
    \end{center}
  \end{frame}

  \begin{frame}[fragile]{Tujuan Mini Projek}
    \begin{itemize}
        \item Membuat fungsi untuk mengecek tangan kemenangan, bertipe seperti
\begin{lstlisting}[language = Haskell]
    agariCheck :: Hand -> Bool
\end{lstlisting}
        \item Membuat fungsi untuk menghitung shanten number, atau seberapa jauh hand dari agari, bertipe seperti
\begin{lstlisting}
    shantenCalculator :: Hand -> Int
\end{lstlisting}
        \item Membuat fungsi untuk menghitung skor yang diperoleh dari agari, bertipe seperti
\begin{lstlisting}
    scoreCalculator :: Hand -> Maybe Int
\end{lstlisting}
    \end{itemize}
  \end{frame}

  \section{Target Progres}
  
  \begin{frame}[allowframebreaks]{Target Progres per Pekan}
    \begin{itemize}
      \item Target Pekan 1: 
      \begin{itemize}
        \item Menyusun tipe data untuk Suit, Tile, Hand, Meld, dan Agari.
        \item Membuat fungsi untuk mencari pair dan meld dari suatu hand.
      \end{itemize}
      \item Target Pekan 2:
      \begin{itemize}
        \item Membuat fungsi validasi tangan kemenangan untuk hand standar, yaitu \(4 \) meld dan \(1 \) pair.
        \item Membuat fungsi validasi tangan kemenangan untuk hand tidak standar, yaitu Chii Toitsu (7 pair) dan Kokushi Musou (\(13 + 1\) tile honor dan terminal).
        
        Fungsi bertipe seperti \lstinline|agariCheck :: Hand -> Maybe Partition|
        dengan \lstinline|Just Partition| menandakan hand yang valid beserta partisinya, dan \lstinline|Nothing| jika hand tidak valid.
      \end{itemize}
      \framebreak
      \item Target Pekan 3:
      \begin{itemize}
        \item Mengidentifikasi semua kandidat set dalam suatu hand: Meld, pair, dan meld yang tidak lengkap.
        \item Mengimplementasi fungsi untuk menghitung nilai shanten, seperti \lstinline|calculateShanten :: Hand -> Int|.
        \item Mengadaptasi algoritma untuk menangani hand yang tidak standar.
      \end{itemize}
      \framebreak

      \item Target Pekan 4:
      \begin{itemize}
        \item Mendefinisikan tipe data untuk merepresentasikan semua Yaku yang mungkin.
        \item Mengimplementasikan fungsi untuk mengecek yaku dari tangan, bertipe seperti \lstinline|isTanyao :: Hand -> Bool|.
        \item Membuat fungsi yang menerima sebuah hand dan mengembalikan daftar semua Yaku yang terpenuhi.
        \item Membuat fungsi untuk menghitung \textit{Fu} (poin minor) dari bentuk hand.
        \item membuat fungsi untuk menghitung skor yang didapat berdasarkan Yaku (Han) dan Fu dari tangan.
      \end{itemize}
      \framebreak
      \item Target Pekan 5:
      \begin{itemize}
        \item Melakukan testing terhadap fungsi yang telah dibuat.
        \item Mulai membangun Command-Line Interface sederhana.
      \end{itemize}
      \item Target Pekan 6:
      \begin{itemize}
        \item Melanjutkan testing dan CLI, serta finalisasi.
        \item Membuat dokumentasi.
      \end{itemize}
    \end{itemize}
  \end{frame}

  \section{Aspek Fungsional}

  \begin{frame}{Aspek Fungsional}
    \begin{itemize}
      \item \textbf{Immutability \& Pure Function}
      
      Fungsi seperti \lstinline|calculateShanten| selalu memberikan hasil yang sama untuk input yang sama.
      \item \textbf{Typeclass, ADT, \& Pattern Matching}
      
      Untuk memodelkan semua entitas dan aturan permainan secara type-safe.
      \item \textbf{Monad}
      
      Mencakup \lstinline|State Monad| untuk mengelola state internal, \lstinline|Reader Monad| yang menyediakan konteks permainan, dan \lstinline|IO Monad|.
      \item \textbf{Higher-Order Function}
      
      Seperti \lstinline|map|, \lstinline|filter|, \lstinline|foldl| yang digunakan untuk memproses \lstinline|Hand|.
    \end{itemize}
  \end{frame}
\end{document}