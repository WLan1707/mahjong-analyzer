\documentclass{beamer}
\usetheme{metropolis}           % Use metropolis theme
\usepackage{graphicx}
\usepackage[dvipsnames]{xcolor}
\usepackage{listings}
\usepackage{todonotes}

\setuptodonotes{inline}
\todostyle{done}{color=green!40}
\todostyle{almostdone}{color=Goldenrod!70}

\usepackage{hyperref}

\hypersetup{
  colorlinks=true,      % aktifkan warna
  linkcolor=blue!70!black,       % warna untuk link internal (misal ke section)
  citecolor=red,        % warna untuk citation
  urlcolor=blue!70!black,        % warna untuk hyperlink / URL
}

\title{Mahjong Analyzer}
\subtitle{Progress 5}
\date{\today}
\author{Muhammad Jilan Wicaksono}
\institute{Universitas Indonesia}

\lstset{
  language=Haskell,            % Bahasa default
  basicstyle=\ttfamily\footnotesize,  % Font dan ukuran
  keywordstyle=\color{blue!70!black},       
  commentstyle=\color{gray!70!black},       
  stringstyle=\color{orange!70!black},      
  numbers=left,                % Tampilkan nomor baris di kiri
  numberstyle=\tiny\color{gray}, % Style nomor baris
  breaklines=true,             % Otomatis pindah baris jika terlalu panjang
  frame=single,                % Beri bingkai
}

\begin{document}
\maketitle

% \section{Target dan Pencapaian}
\begin{frame}[fragile,allowframebreaks]{Target dan Pencapaian}
    \todo[almostdone]{\textbf{Membuat Suite Test menggunakan QuickCheck:} \hfill 70\%}
    \begin{itemize}
      \item Membuat generator dari QuickCheck untuk Data Type Tile.
      \item Membuat generator untuk Data Type AgariHand dan Meld berdasarkan genTile.
      \item Membuat daftar properti yang secara matematis pasti terpenuhi oleh fungsi-fungsi yang sudah dibuat.
    \end{itemize}
    \todo[almostdone]{\textbf{Menulis README:} \hfill 30\%}
    \begin{itemize}
      \item Berisi penjelasan singkat algoritma dari penghitung shanten dan cara kerja memoTree.
    \end{itemize}

    \framebreak
    \begin{columns}
      \begin{column}{.5\linewidth}
        
        \begin{center}
          \includegraphics[width = \linewidth]{images/testGenTile.png}
        \end{center}
    
      \end{column}
      \begin{column}{.5\linewidth}
        
        \begin{center}
          \includegraphics[width = \linewidth]{images/testGenMeld.png}
        \end{center}
      \end{column}
    \end{columns}

    \framebreak

    \begin{center}
      \includegraphics[width = .75\linewidth]{images/testGenAgariHand.png}
    \end{center}

    \framebreak
    \begin{columns}
  
      \begin{column}{.5\linewidth}
        \begin{center}
          \includegraphics[width = \linewidth]{images/propFindPartition.png}
        \end{center}
      \end{column}

      \begin{column}{.5\linewidth}
        \begin{center}
          \includegraphics[width = \linewidth]{images/propFu.png}
        \end{center}
      \end{column}
    \end{columns}

    \framebreak

    \begin{center}
      \includegraphics[width = .75\linewidth]{images/propInvariant.png}
    \end{center}
    \framebreak

    \begin{center}
      \includegraphics[width = \linewidth]{images/testRes.png}
    \end{center}
    
\end{frame}

% \section{Link Commit}
\begin{frame}{Link Commit}
  \href{https://github.com/WLan1707/mahjong-analyzer/commit/7de788b591968764c895ec04c64a1afe44b4b6be}{Link Commit}
    \begin{itemize}
        \item \href{https://github.com/WLan1707/mahjong-analyzer/commit/7de788b591968764c895ec04c64a1afe44b4b6be}{https://github.com/WLan1707/mahjong-analyzer/commit/7de788b591968764c895ec04c64a1afe44b4b6be}
    \end{itemize}
\end{frame}

% \section{Lesson Learned}
\begin{frame}[fragile, allowframebreaks]{Lesson Learned: Memoization Using Tree}
  Secara konsep mirip seperti Fibonacci, jadi akan digunakan contoh ini saja:
  \[F(0) = 1, F(1) = 1, F(n) = F(n-1) + F(n-2)\]
  Tanpa memoization, dapat memanggil

  \lstinline|fix fib 30 == 1346269|

  \begin{center}
    \includegraphics[width = .7\linewidth]{images/no_memo.png}
  \end{center}

  \framebreak

  Dengan list, dapat dipanggil 
  
  \lstinline|faster_f 3000 == 9001823511092631249|
  \begin{center}
    \includegraphics[width = .5\linewidth]{images/memoList.png}
  \end{center}
  Jika diperhatikan, \lstinline|faster_f| memanfaatkan \lstinline|fmap| dari Functor List dengan fungsi lookup \lstinline|(!!)| sebesar \(O(n)\).

  dapat dimanfaatkan Functor lain dengan fungsi lookup sebesar \(O(\log n)\), yaitu tree.

  \framebreak
  \begin{columns}[T, onlytextwidth]
    \begin{column}{.5\textwidth}
      \begin{center}
        \includegraphics[width = .7\linewidth]{images/treeVisualization.png}
      \end{center}
    \end{column}
    \begin{column}{.5\textwidth}
      \includegraphics[width = .7\linewidth]{images/memoTree.png}.
    \end{column}
  \end{columns}
  \lstinline|f_tree| dibuat dulu secara lazy, ketika \lstinline|fastest_f n| dipanggil, hanya perlu melihat node ke \((n - 1)\) dan \((n - 2)\) dari \lstinline|f_tree| menggunakan index, dan keduanya juga hanya melihat kode sebelumnya saja juga, dan seterusnya. 

  \framebreak

  Dengan ini, dapat dipanggil bilangan fibonacci ke \(70000\) secara cepat.
  \begin{center}
    \includegraphics[width = \linewidth]{images/fastest_f 70000.png}
  \end{center}
  \framebreak

  Penerapan pada perhitungan shanten
  \begin{center}
    \includegraphics[width = \linewidth]{images/shantenHC.png}
  \end{center}

  \framebreak

  \begin{center}
    \includegraphics[width = .8\linewidth]{images/shantenInt.png}
  \end{center}

  \begin{center}
    \includegraphics[width = .5\linewidth]{images/memoShanten.png}
  \end{center}

\end{frame}

\begin{frame}[fragile, allowframebreaks]{Lesson Learned: QuickCheck}
  Haskell sudah mempunyai cara untuk membangun domain dari suatu fungsi secara otomatis, yang bisa digunakan untuk melakukan pengecekan suatu sifat dari fungsi itu.

  Meskipun secara otomatis, perlu diberikan argumen agar domain tersebut membuat fungsi itu terdefinisi dengan baik di sana.

  Pengecekannya juga hanya berupa sifat, yang harus bisa dibuktikan secara matematis (atau dari aturannya) tanpa program.

  Sehingga harus berhati-hati dalam pendefinisian domain dan sifat yang ingin diperiksa.

  \framebreak

  Kelas Arbitrary juga memiliki tipe Gen, yang merupakan monad dan pastinya aplikatif, sehingga ada dua cara penulisan generator:
  
  \begin{enumerate}
    \item do-notation jika generator bergantung antara variable

    \begin{center}
      \includegraphics[width = .6\linewidth]{images/genSeq.png}
    \end{center}

    \item Applicative untuk generator yang independen
    
    \begin{center}
      \includegraphics[width = \linewidth]{images/genStd.png}
    \end{center}
  \end{enumerate}
  Ini membuat kode generator lebih ekspresif, serta jangan paksakan penggunaan suatu notasi jika ada yang lebih bagus.
\end{frame}
\end{document}