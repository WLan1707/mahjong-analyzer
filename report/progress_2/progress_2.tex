\documentclass{beamer}
\usetheme{metropolis}           % Use metropolis theme
\usepackage{graphicx}
\usepackage[dvipsnames]{xcolor}
\usepackage{listings}
\usepackage{todonotes}

\setuptodonotes{inline}
\todostyle{done}{color=green!40}
\todostyle{almostdone}{color=Goldenrod!70}

\usepackage{hyperref}

\hypersetup{
  colorlinks=true,      % aktifkan warna
  linkcolor=blue!70!black,       % warna untuk link internal (misal ke section)
  citecolor=red,        % warna untuk citation
  urlcolor=blue!70!black,        % warna untuk hyperlink / URL
}

\title{Mahjong Analyzer}
\subtitle{Progress 2}
\date{\today}
\author{Muhammad Jilan Wicaksono}
\institute{Universitas Indonesia}

\lstset{
  language=Haskell,            % Bahasa default
  basicstyle=\ttfamily\footnotesize,  % Font dan ukuran
  keywordstyle=\color{blue!70!black},       
  commentstyle=\color{gray!70!black},       
  stringstyle=\color{orange!70!black},      
  numbers=left,                % Tampilkan nomor baris di kiri
  numberstyle=\tiny\color{gray}, % Style nomor baris
  breaklines=true,             % Otomatis pindah baris jika terlalu panjang
  frame=single,                % Beri bingkai
}

\begin{document}
\maketitle

% \section{Target dan Pencapaian}
\begin{frame}{Target dan Pencapaian}
    \begin{enumerate}
        \item \todo[done]{\textbf{Menyusun tipe data:} \hfill 100\%}
        untuk Suit, Tile, Hand, Meld, dan Agari.
        \item \todo[done]{\textbf{Membuat fungsi utilitas:} \hfill 100\%}
        untuk mencari Pair dan Triplet, serta membuat parser sederhana.
        \item \todo[almostdone]{\textbf{Membuat fungsi validasi kemenangan:} \hfill 75\%}
        baik standar maupun non-standar. 
        
        Target bertipe \lstinline|agariCheck :: Hand -> Maybe Partition|

        tetapi masih bertipe \lstinline|agariCheck :: Hand -> Bool|
    \end{enumerate}
\end{frame}

% \section{Link Commit}
\begin{frame}{Link Commit}
    \begin{itemize}
        \item Untuk \href{https://github.com/WLan1707/mahjong-analyzer/commit/543d4e09ed201b961832f01c22b7852d8a0c3fa0}{agariCheck}: Membuat fungsi validasi tangan sederhana.
        \item Untuk \href{https://github.com/WLan1707/mahjong-analyzer/commit/df2459565765f7a16bf2800e2b3bbc6123f348bf}{fungsi utilitas}: Membuat ADT untuk entitas permainan serta menambahkan parser sederhana.
    \end{itemize}
\end{frame}

% \section{Lesson Learned}
\begin{frame}{Lesson Learned}
    \begin{enumerate}
        \item ADT untuk \lstinline|Tile, Pair, Meld| agar type-safe.
        \item Higher-Order Function untuk fungsi \lstinline|removeTiles| menggunakan \lstinline|removeOneTile|.
        \item Recursion dalam pemanggilan \lstinline|agariCheck|.
        \item \lstinline|Either Monad| pada parser untuk error handling.
    \end{enumerate}
\end{frame}

\begin{frame}[fragile, allowframebreaks]{Lesson Learned: ADT}
    \begin{lstlisting}
data Suit = Manzu | Pinzu | Souzu | Honor
    deriving (Show, Eq, Ord, Enum, Bounded)

data Tile = Tile Suit Int
    deriving (Show, Eq)

instance Ord Tile where
    compare :: Tile -> Tile -> Ordering
    compare (Tile s1 n1) (Tile s2 n2)
        | s1 == s2   = compare n1 n2
        | otherwise  = compare s1 s2
    \end{lstlisting}
    \framebreak
    \begin{lstlisting}
data Meld 
    = Sequence Tile         -- Sequence M4 : M4, M5, M6
    | Triplet Tile          -- Triplet Z3 : Z3, Z3, Z3
    deriving (Show)

data PMeld 
    = MissingMiddle Tile    -- MissingMiddle P4 : ada P4 dan P6
    | MissingOut Tile       -- MissingOut S2 : ada S2 dan S3
    deriving (Show)

newtype Pair = Pair Tile  -- Pair P3 : P3, P3
    \end{lstlisting}
\end{frame}

\begin{frame}[allowframebreaks, fragile]{Lesson Learned: Higher-Order Function}
    \begin{lstlisting}
removeOneTile :: Tile -> HandCount -> HandCount
removeOneTile =
    let updateFunc c = if c > 1 then Just (c - 1) else Nothing
        in Map.update updateFunc 

removePair :: Pair -> HandCount -> HandCount
removePair (Pair tile) hc =
    let tilesToRemove = replicate 2 tile
        in foldl' (flip removeOneTile) hc tilesToRemove
    \end{lstlisting}
    \framebreak
    \begin{lstlisting}
removeSequence :: Meld -> HandCount -> HandCount
removeSequence (Sequence (Tile suit num)) handCount = 
    let tilesToAdjust = [Tile suit num, Tile suit (num + 1), Tile suit (num + 2)]

    in foldl' (flip removeOneTile) handCount tilesToAdjust

removeTriplet :: Meld -> HandCount -> HandCount
removeTriplet (Triplet tile) handCount = 
    let tilesToAdjust = replicate 3 tile

    in foldl' (flip removeOneTile) handCount tilesToAdjust 
    \end{lstlisting}
\end{frame}

\begin{frame}[fragile, allowframebreaks]{Lesson Learned: Recursion}
    \begin{lstlisting}
checkMeld :: HandCount -> Int -> Bool
checkMeld hc meldToFind
  | meldToFind == 0 = Map.null hc
  | otherwise =
    case Map.minViewWithKey hc of
      Nothing -> False

      Just ((firstTile, count), _) ->
                
        let tryTriplet = 
          (count >= 3) && 
          let m = Triplet firstTile
            hc' = removeTriplet m hc 
              in checkMeld hc' (meldToFind - 1)

          trySequence = 
            isSequence firstTile hc &&
            let m = Sequence firstTile
              hc' = removeSequence m hc
            in checkMeld hc' (meldToFind - 1)

        in tryTriplet || trySequence
    \end{lstlisting}
\end{frame}

\begin{frame}[fragile, allowframebreaks]{Lesson Learned: Monad}
    \begin{lstlisting}
parseHand :: String -> Either String Hand
parseHand "" = Right []  -- Base case: tangan kosong
parseHand str =
  -- Pisahkan grup angka di depan
  -- span isDigit "123m456p" -> ("123", "m456p")
  let (nums, rest) = span isDigit str
  in
    if null nums then
      Left "Input tidak valid: Diharapkan ada angka."
    else if null rest then
      Left "Input tidak valid: Diharapkan ada suit (m,p,s,z) setelah angka."
    else
      let suitChar = head rest
        remainingString = tail rest
      in
        case charToSuit suitChar of
          Left err   -> Left err
          Right suit ->
            -- Buat tiles untuk bagian ini
            let currentTiles = stringToTiles nums suit
            -- Panggil rekursi untuk sisa string
            in (currentTiles ++) <$> parseHand remainingString
    \end{lstlisting}
    
\end{frame}
\end{document}