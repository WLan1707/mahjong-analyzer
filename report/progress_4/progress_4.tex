\documentclass{beamer}
\usetheme{metropolis}           % Use metropolis theme
\usepackage{graphicx}
\usepackage[dvipsnames]{xcolor}
\usepackage{listings}
\usepackage{todonotes}

\setuptodonotes{inline}
\todostyle{done}{color=green!40}
\todostyle{almostdone}{color=Goldenrod!70}

\usepackage{hyperref}

\hypersetup{
  colorlinks=true,      % aktifkan warna
  linkcolor=blue!70!black,       % warna untuk link internal (misal ke section)
  citecolor=red,        % warna untuk citation
  urlcolor=blue!70!black,        % warna untuk hyperlink / URL
}

\title{Mahjong Analyzer}
\subtitle{Progress 4}
\date{\today}
\author{Muhammad Jilan Wicaksono}
\institute{Universitas Indonesia}

\lstset{
  language=Haskell,            % Bahasa default
  basicstyle=\ttfamily\footnotesize,  % Font dan ukuran
  keywordstyle=\color{blue!70!black},       
  commentstyle=\color{gray!70!black},       
  stringstyle=\color{orange!70!black},      
  numbers=left,                % Tampilkan nomor baris di kiri
  numberstyle=\tiny\color{gray}, % Style nomor baris
  breaklines=true,             % Otomatis pindah baris jika terlalu panjang
  frame=single,                % Beri bingkai
}

\begin{document}
\maketitle

% \section{Target dan Pencapaian}
\begin{frame}[fragile,allowframebreaks]{Target dan Pencapaian}
    \todo[almostdone]{\textbf{Membuat fungsi pencari partisi} \hfill 90\%}
    \begin{itemize}
      \item Memodifikasi fungsi validasi kemenangan menjadi pencari partisi yang bertipe bisa mengeluarkan semua partisi yang menang.
      \item Belum sempat mengoptimasi lagi seperti fungsi shanten.
    \end{itemize}
    \todo[almostdone]{\textbf{Mengimplementasikan fungsi untuk mengecek yaku:} \hfill 70\%}
    \begin{itemize}
      \item Mengimplementasikan fungsi yang menerima partisi dan mengeluarkan list semua yaku yang terpenuhi.
      \item Baru mengimplementasikan \(17 \) dari sekitar \(26 \) pola yaku yang ada. Kebanyakan adalah pola edge case yang langka.
    \end{itemize}

    \framebreak
    \todo[almostdone]{\textbf{Membuat fungsi untuk menghitung skor:} \hfill 80\%}
    \begin{itemize}
      \item Mengimplementasikan fungsi yang menerima partisi dan mengeluarkan nilai Fu dari tangan, yaitu skor minor.
      \item Menghitung total skor yang diperoleh berdasarkan pola yang dibuat dan nilai dari Fu, serta informasi eksternal seperti apakah menjadi dealer.
    \end{itemize}

    \framebreak
    Yaku Triple Mixed Sequence: Memuat 3 sequence dengan nilai sama tapi menggunakan 3 suit berbeda
    \begin{columns}
      \begin{column}{.5\linewidth}
        \begin{center}
          \includegraphics[width = \linewidth]{images/mixedTripleSequenceEx.png}
        \end{center}
        \begin{center}
          \includegraphics[width = \linewidth]{images/mixedTripleSequenceRes.png}
        \end{center}
      \end{column}
      \begin{column}{.5\linewidth}
        \begin{center}
          \includegraphics[width = \linewidth]{images/mixedTripleSequenceCode.png}
        \end{center}
      \end{column}
    \end{columns}

    \framebreak
    Yaku Pure Straight: Memuat sequence (1,2,3), (4,5,6), (7,8,9) dengan suit yang sama.
    \begin{columns}
      \begin{column}{.5\linewidth}
        \begin{center}
          \includegraphics[width = \linewidth]{images/pureStraightEx.png}
        \end{center}
        \begin{center}
          \includegraphics[width = \linewidth]{images/pureStraightRes.png}
        \end{center}
      \end{column}
      \begin{column}{.5\linewidth}
        \begin{center}
          \includegraphics[width = \linewidth]{images/pureStraightCode.png}
        \end{center}
      \end{column}
    \end{columns}
    
\end{frame}

% \section{Link Commit}
\begin{frame}{Link Commit}
  \href{https://github.com/WLan1707/mahjong-analyzer/commit/e948dcd73627e5b8a2a4945314042c5f6dc07741}{Link Commit}
    \begin{itemize}
        \item \href{https://github.com/WLan1707/mahjong-analyzer/commit/e948dcd73627e5b8a2a4945314042c5f6dc07741}{https://github.com/WLan1707/mahjong-analyzer/commit/e948dcd73627e5b8a2a4945314042c5f6dc07741}
    \end{itemize}
\end{frame}

% \section{Lesson Learned}
\begin{frame}[fragile, allowframebreaks]{Lesson Learned: Memoization Using Tree}
  Secara konsep mirip seperti Fibonacci, jadi akan digunakan contoh ini saja:
  \[F(0) = 1, F(1) = 1, F(n) = F(n-1) + F(n-2)\]
  Tanpa memoization, dapat memanggil

  \lstinline|fix fib 30 == 1346269|

  \begin{center}
    \includegraphics[width = .7\linewidth]{images/no_memo.png}
  \end{center}

  \framebreak

  Dengan list, dapat dipanggil 
  
  \lstinline|faster_f 3000 == 9001823511092631249|
  \begin{center}
    \includegraphics[width = .6\linewidth]{images/memoList.png}
  \end{center}
  Seperti yang dipelajari di kelas

  \lstinline|fib' = 1 : 1 : zipWith (+) fib' (tail fib')|

  Jika diperhatikan, \lstinline|faster_f| memanfaatkan \lstinline|fmap| dari Functor List dengan fungsi lookup \lstinline|(!!)| sebesar \(O(n)\).

  \framebreak
  \begin{columns}[T, onlytextwidth]
    \begin{column}{.5\textwidth}
      \begin{center}
        \includegraphics[width = .7\linewidth]{images/treeVisualization.png}
      \end{center}
    \end{column}
    \begin{column}{.5\textwidth}
      \includegraphics[width = .7\linewidth]{images/memoTree.png}.
    \end{column}
  \end{columns}
  \lstinline|f_tree| dibuat dulu secara lazy, ketika \lstinline|fastest_f n| dipanggil, hanya perlu melihat node ke \((n - 1)\) dan \((n - 2)\) dari \lstinline|f_tree| menggunakan index, dan keduanya juga hanya melihat kode sebelumnya saja juga, dan seterusnya.

  \framebreak

  Dengan ini, dapat dipanggil bilangan fibonacci ke \(70000\) secara cepat.
  \begin{center}
    \includegraphics[width = \linewidth]{images/fastest_f 70000.png}
  \end{center}
  \framebreak

\end{frame}

% \begin{frame}[fragile, allowframebreaks]{Lesson Learned: Kompleksitas Tipe Data}
  
% \end{frame}
\end{document}