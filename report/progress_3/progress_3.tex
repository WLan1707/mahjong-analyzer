\documentclass{beamer}
\usetheme{metropolis}           % Use metropolis theme
\usepackage{graphicx}
\usepackage[dvipsnames]{xcolor}
\usepackage{listings}
\usepackage{todonotes}

\setuptodonotes{inline}
\todostyle{done}{color=green!40}
\todostyle{almostdone}{color=Goldenrod!70}

\usepackage{hyperref}

\hypersetup{
  colorlinks=true,      % aktifkan warna
  linkcolor=blue!70!black,       % warna untuk link internal (misal ke section)
  citecolor=red,        % warna untuk citation
  urlcolor=blue!70!black,        % warna untuk hyperlink / URL
}

\title{Mahjong Analyzer}
\subtitle{Progress 3}
\date{\today}
\author{Muhammad Jilan Wicaksono}
\institute{Universitas Indonesia}

\lstset{
  language=Haskell,            % Bahasa default
  basicstyle=\ttfamily\footnotesize,  % Font dan ukuran
  keywordstyle=\color{blue!70!black},       
  commentstyle=\color{gray!70!black},       
  stringstyle=\color{orange!70!black},      
  numbers=left,                % Tampilkan nomor baris di kiri
  numberstyle=\tiny\color{gray}, % Style nomor baris
  breaklines=true,             % Otomatis pindah baris jika terlalu panjang
  frame=single,                % Beri bingkai
}

\begin{document}
\maketitle

% \section{Target dan Pencapaian}
\begin{frame}{Target dan Pencapaian}
    \todo[done]{\textbf{Mengimplementasikan fungsi penghitung shanten:} \hfill 100\%}
    \begin{itemize}
      \item Mengidentifikasi apakah suatu tile merupakan bagian dari meld, pair, ataupun pmeld.
      \item Dapat menghitung meskipun ada meld yang sudah terbuka.
      \item Dapat menghitung untuk hand yang tidak standar.
      \item Optimisasi menggunakan memoization.
    \end{itemize}
\end{frame}

% \section{Link Commit}
\begin{frame}{Link Commit}
    \begin{itemize}
        \item \href{https://github.com/WLan1707/mahjong-analyzer/commit/e691aec4f02b99c42f568873b3faacf42b7e9a44}{https://github.com/WLan1707/mahjong-analyzer/commit/e691aec4f02b99c42f568873b3faacf42b7e9a44}
        \item \href{https://github.com/WLan1707/mahjong-analyzer/commit/9754714ae62ab10469f3a9ef9a12bba3aafe9848}{https://github.com/WLan1707/mahjong-analyzer/commit/9754714ae62ab10469f3a9ef9a12bba3aafe9848}
    \end{itemize}
\end{frame}

% \section{Lesson Learned}
\begin{frame}[fragile, allowframebreaks]{Lesson Learned: Lazy Evaluation \& Memoization}
  \begin{columns}[T, onlytextwidth]
    \begin{column}{.5\textwidth}
      \begin{itemize}
        \item Kelompokkan Hand berdasarkan suit, lalu ubah menjadi bilangan asli.
      \end{itemize}
    \end{column}
    
    \begin{column}{.5\textwidth}
      \includegraphics[width = \linewidth]{images/extractSuit.png}
    \end{column}
  \end{columns}
  Perhitungan shanten kini dilakukan per suit, dan hasil per-suit dapat dimemoisasi menggunakan struktur lazy agar evaluasi setiap state hanya terjadi sekali.

  \framebreak

  Masalah: jika menggunakan list, indexing dilakukan secara linier.
  \begin{columns}[T, onlytextwidth]
    \begin{column}{.5\textwidth}
      \begin{itemize}
        \item Buat infinite tree bilangan asli
        \item Karena fungsi bertipe \lstinline|Int -> Int|, gunakan functor dari tree untuk mengubah fungsi shanten menjadi infinite tree juga, dengan node ke \(i \) bernilai \lstinline|shanten i|.
        \item Karena disimpan dalam tree, mencari node \(n \) cukup dalam waktu \(O(\log n)\).
      \end{itemize}
    \end{column}
    \begin{column}{.4\textwidth}
      \includegraphics[width = \linewidth]{images/tree.png}

      \includegraphics[width = \linewidth]{images/memoTree.png}
    \end{column}
  \end{columns}

\end{frame}
\end{document}